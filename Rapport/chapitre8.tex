\chapter{Les apports du projet}
\section{Les difficultés rencontrés}
Au cours de ce projet nous avons rencontré plusieurs difficultés. La première était de taille puisqu'au commencement du projet nous n'avions aucune expérience en développement Android. Nous avons donc du passer par une phase d'apprentissage assez longue des différents composants de l'API.\bigskip

La seconde difficulté réside dans l'algorithme de comparaison de deux trajets. Nous avons réussi à définir un algorithme qui traite les cas simples lorsque les deux trajectoires sont relativement proches. Mais lorsque celles ci diffèrent d'avantage il devient très complexe de traiter tous les cas. Nous n'avons donc pas pu mettre en production dans l'application la comparaison en temps réel de deux trajets.\bigskip

Enfin pour tester notre application nous ne disposions pas de terminal mobile avec une puce GPS suffisamment puissante. En fonction de la marque et de la qualité du mobile les fabricants insèrent des puces GPS plus ou moins performantes dans leurs produits. Nous avons donc du emprunter un smartphone à nos collègues lorsque c'était possible pour tester l'acquisition des coordonnées GPS.\bigskip

\section{Apprentissage}
