\chapter{Les apports du projet}
\section{Les difficultés rencontrés}
Au cours de ce projet nous avons rencontré plusieurs difficultés. La première était de taille puisqu'au commencement du projet nous n'avions aucune expérience en développement Android. Même si nous avions de bonnes connaissances en Java, nous avons dû apprendre le fonctionnement des différents composants d'Android.\bigskip

La seconde difficulté réside dans l'algorithme de comparaison de deux trajets. Nous avons réussi à définir un algorithme qui traite les cas simples lorsque les deux trajectoires sont relativement proches. Mais il devient très complexe dès lors où il faut traiter tous les cas. Et même si nous n'étions pas loin de réussir, nous n'avons donc pas pu ajouter la comparaison de deux trajets en temps réel.\bigskip

Enfin, pour tester notre application nous ne disposions pas d'appareil avec une puce GPS suffisamment puissante pour capter rapidement de GPS. On se retrouvait souvent avec des acquisitions de données très hétérogènes pour un même trajet.\bigskip

\section{Apprentissage}
Au cours de ce projet, nous avons pu apprendre beaucoup de choses en matière de développement pour mobile. Les limitations apportées par la machine sont assez rares lorsque l'on développe des applications pour ordinateur. Sur mobile, même si elle est intégrée aux composants Android, la notion de développement parallèle est beaucoup plus présente que sur PC. La gestion de l'autonomie de l'appareil est également très importante et n'apparait pas du tout sur nos machines traditionnelles.
\bigskip
L'une des parties les plus intéressantes de ce projet a été de se confronter au problème de comparaison de trajets. Même si il est frustrant de ne pas avoir réussit à l'implémenter, la confrontation au problème fut très enrichissante. Il est rare à notre niveau d'avoir à ce pencher sur des problèmes qui paraissent simples mais qui sont beaucoup plus complexes.