\part{Analyse et implémentations}
\chapter{Les technologies utilisées}
\section{Le choix du langage}
\subsection{Javascript}
Nous avons réalisé une étude sur les différentes technologies que nous pouvions employées pour réaliser ce projet. En effet, avec les technologies Javascript émergentes, l'utilisation de Cordova et Phonegap permettait de réaliser une application compatible avec l'ensemble des smartphones présent sur le marché actuel. Malheureusement, ces technologies sont plutôt orientées sur des technologies web. Le développement du code avec ces technologies nécessite une compilation afin d'être lisible par le smartphone sur lequel il est déployé. 
\subsection{Java}
L'avantage avec le développement sous Android vient du fait que le langage utilisé n'est autre que du Java. De plus, la compilation du langage est optimisée en fonction du téléphone sur lequel il est déployé. De ce fait, l’exécution d'une application est plus performante si elle est conçue dans le langage natif au smartphone(Java pour Android, Objectif-C pour IOS). Notre application utilisant le GPS du smartphone, nous avons opté pour un développement en Java afin d'optimiser les connexions avec le satellite servant à géolocaliser le téléphone.
\section{La base de données}
