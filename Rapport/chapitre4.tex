\chapter{Les technologies utilisées}
\section{Le choix du langage}

\subsection{Javascript}
Nous avons réalisé une étude sur les différentes technologies que nous pouvions employer pour réaliser ce projet. En effet, avec les technologies Javascript émergentes, l'utilisation de frameworks tel que Cordova ou Phonegap permet de réaliser des applications compatibles avec l'ensemble des smartphones. Malheureusement, ces frameworks sont plutôt orientées sur des technologies web et donc, le développement du code avec ces technologies nécessite une compilation afin d'être lisible par le smartphone sur lequel il est déployé. 

\subsection{Java}
Le développement sous Android se fait dans un langage très populaire dans les systèmes embarqués, le Java. De plus, la compilation du langage est optimisée en fonction de l’appareil sur lequel il est déployé. De ce fait, l’exécution d'une application est plus performante si elle est conçue dans le langage natif du smartphone(Java pour Android, Objectif-C pour iOS). Notre application utilisant le GPS du smartphone, nous avons opté pour un développement en Java afin d'optimiser les connexions avec le satellite servant à géolocaliser le téléphone. Android étant mis à disposition par Google, il existe une documentation extrêmement fournie par l'API Google\footnote{\href{http://developer.android.com/index.html}{Google API}}

\section{Le stockage de données}
Pour un bon fonctionnement, l'application doit stocker des informations concernant les parcours et les trajets. 

\subsection{SQLite}
Sous Android, le stockage des données se fait via une base de données SQLite. SQLite est une bibliothèque proposant un moteur de base de données relationnelles utilisant le langage SQL. Elle est plus légère que ses homologues MySQL et PostgreSQL car elle n'intègre pas le schéma habituel Client-Serveur. En effet, l'ensemble des données est stocké dans un fichier indépendant d'Android et directement intégré à l'application.

\subsection{Le format GPX}
Le GPX est un format XML conçu spécialement pour représenter un ensemble de points GPS ayant une latitude et une longitude à un instant donné. L'utilisation de fichiers GPX est très intéressante dans notre projet dans la mesure où le GPS d'Android nous fournit des points devant être conservés mais sans encombrer la base. En effet, SQLite n'est pas très adapté au stockage d'autant de données. De plus, cette séparation permet d'extraire facilement les données liées à la représentation d'un trajet.

\begin{xml}[Exemple de fichier GPX]
<?xml version="1.0" encoding="UTF-8" standalone="no" ?>
<gpx xmlns="http://www.topografix.com/GPX/1/1" creator="MapSource 6.9.2" version="1.1" xmlns:xsi="http://www.w3.org/2001/XMLSchema-instance" xsi:schemaLocation="http://www.topografix.com/GPX/1/1 http://www.topografix.com/GPX/1/1/gpx.xsd">
  <metadata>
    <time>2011-04-13T15:58:51Z</time>
    <bounds maxlat="49.746009" maxlon="-1.372054" minlat="49.644963" minlon="-1.925924"/>
  </metadata>
  <trk>
    <trkseg>
      <trkpt lat="49.645132" lon="-1.620045">
        <ele>-46.486572</ele>
        <time>2011-04-06T08:47:37Z</time>
      </trkpt>
     ...
      <trkpt lat="49.645703" lon="-1.619700">
        <ele>-3.227295</ele>
        <time>2011-04-06T08:49:37Z</time>
    </trkseg>
  </trk>
</gpx>
\end{xml}

\subsection{L'analyseur syntaxique}
L'utilisation de fichier GPX implique forcément un système de lecture et d'écriture de fichier XML (\textit{parser}). En Java, il existe deux grandes familles de parseurs XML, les parseurs utilisant SAX et les parseurs utilisant DOM.

\subsubsection{Les parseurs DOM}
Les parseurs DOM analysent la structure entière d'un document. Ils stockent en mémoire l'ensemble des balises XML (le DOM) afin de pouvoir retrouver n'importe quel élément. L'inconvénient de ce parseur vient du fait qu'un document doit être enregistré dans la mémoire vive de l'appareil. De ce fait, l'utilisation de ce parseur est très gourmand en terme de ressources.

\subsubsection{Les parseurs SAX}
Les parseurs SAX sont événementiels. En effet, ils analysent le document et déclenchent un événement lorsqu'une balise est construite ou détruite. Cette façon de lire un document est nettement moins coûteuse que la précédente dans la mesure où rien n'est stockée en mémoire. La lecture du document se fait au fur et à mesure. Cette méthode est beaucoup plus adaptée pour notre projet dans la mesure où les fichiers GPX analysés peuvent être volumineux et les appareils n'ont pas beaucoup de mémoire. 
