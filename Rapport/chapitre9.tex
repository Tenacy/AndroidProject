\chapter{Limitations et perspectives}
\section{Les limitations du produit}
Malheureusement, nous n'avons pas eu le temps de développer la totalité des fonctionnalités nécessaires pour faire de notre application un "coach" mobile. En effet, la comparaison de deux trajets n'est pas terminé à cause des difficultés citées précédemment. 


\section{Les perspectives d'avenir}
Beaucoup de fonctionnalités peuvent donc être ajoutées à l'application. En mettant de côté celles citées précédemment on peut facilement imaginer beaucoup de solutions comme par exemple : \bigskip
\begin{itemize}
	\item reconnaître automatiquement si le parcours actuellement réalisé correspond à un parcours déjà effectué
	\item proposer un entraînement progressif en permettant à l'utilisateur de définir des objectifs (en temps ou en pourcentages) par rapport à un trajet précédent
	\item détecter les écarts de trajectoires
	\item synchroniser les données avec un serveur web pour permettre à l'utilisateur de consulter ses données sur le web
	\item exporter les données GPX
	\item créer des graphiques illustrant les performances de l'utilisateur
	\item calculer les calories dépensées au cours d'une courses à partir du type de course (course a pied, marche, cyclisme) et du poids de l'utilisateur
\end{itemize}
