{\Huge{Glossaire}}

\vspace{2cm}

{\bfseries Apache Cordova ou Phonegap:} est un framework open-source développé par la Fondation Apache. Il permet de créer des applications pour différentes plates-formes (Android, Firefox OS, iOS, Ubuntu, Windows 8...) en HTML, CSS et JavaScript.
\bigskip

{\bfseries API (Application Programming Interface):} c'est un ensemble normalisé de classes, de méthodes ou de fonctions qui sert de façade par laquelle un logiciel offre des services à d'autres logiciels.
\bigskip

{\bfseries Base de données:} c’est un ensemble de données stockées qui peuvent être accédées et modifiées via un langage de manipulation de données.
\bigskip

{\bfseries Bibliothèque / Librairie:} désigne un groupe de fonctionnalités dont les caractéristiques sont éditées, et donc à la disposition de différentes applications.
\bigskip

{\bfseries Callback (fonction de rappel):} c'est une fonction qui est passée en argument à une autre fonction. Cette dernière peut alors faire usage de cette fonction de rappel comme de n'importe quelle autre fonction, alors qu'elle ne la connaît pas par avance.
\bigskip

{\bfseries Interface:} c’est la déclaration de signature(s) d'une fonction que toutes les classes héritantes devront implémenter.
\bigskip

{\bfseries JavaScript:} c’est un langage de programmation orienté objet, principalement utilisé dans les pages web.
\bigskip

{\bfseries Modélisation:} Action de décrire un système réel de façon formelle, de façon à pouvoir le manipuler par ordinateur.
\bigskip

{\bfseries Patron de conception:} c'est un arrangement caractéristique de modules, reconnu comme bonne pratique en réponse à un problème de conception d'un logiciel.
\bigskip

{\bfseries SAX (Simple API for XML):} API basée sur un mode événementiel permettant de réagir à des événements (comme le début d'un élément, la fin d'un élément) et de renvoyer le résultat à l'application.
\bigskip

{\bfseries SDK (Software Development Kit)):} c'est un ensemble d'outils permettant aux développeurs d'installer des applications de type défini
\bigskip

 