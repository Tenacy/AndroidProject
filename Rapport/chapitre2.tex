\chapter{Objectifs de l'application}
\section{Objectifs}
Les objectifs de ce projet sont multiples. D'abord il s'agit de réaliser une application mobile pour smartphone Android qui dispose des fonctionnalités suivantes:\bigskip

\begin{itemize}
  \item Enregistrement des coordonnées GPS du trajet
  \item Visualisation du trajet sur une carte 
  \item Suppression d'un trajet 
  \item Analyse des données pour obtenir :
  \begin{itemize}
    \item le temps du trajet
    \item la distance parcourue
    \item la vitesse moyenne
    \item l'allure (temps pour parcourir un kilomètre)
  \end{itemize}
\end{itemize}\bigskip

Le second objectif est certainement le cœur du projet puisque c'est dans celui-ci que réside la nouveauté. Il s'agit de regrouper les trajets identiques pour pouvoir les comparer. Un parcours représente alors un ensemble de trajets qui suivent à peu de chose près la même trajectoire.\bigskip

Avec la mise en place de cette nouvelle structure de données nous pourrons extraire des statistiques intéressantes pour l'utilisateur à savoir:\bigskip

\begin{itemize}
  \item La meilleure performance sur le parcours
  \item Les différentes moyennes du parcours:
  \begin{itemize}
    \item le temps moyen
    \item la distance
    \item la vitesse moyenne
    \item l'allure moyenne (temps pour parcourir un kilomètre)
  \end{itemize}   
\end{itemize}\bigskip

Mais nous pourrons aussi comparer plusieurs trajectoires entre elles pour permettre à l'utilisateur de connaître en temps réel le retard ou l'avance qu'il possède par rapport à un autre trajet. L'utilisateur devra donc pouvoir fixer le trajet à prendre en référence pour chaque parcours.

\section{Diagramme de cas d'utilisation}
A partir de ces besoins, nous avons élaboré un diagramme de cas d'utilisation qui permet de visualiser globalement le comportement de l'application. Il décrit les enchainements d'actions que les différents acteurs peuvent effectuer . Dans notre application il n'existe qu'un seul acteur: l'utilisateur.
\begin{img}
  \includegraphics[scale=0.35]{img/DUC.png}
  \caption{Diagramme de cas d'utilisation de l'application}
\end{img}